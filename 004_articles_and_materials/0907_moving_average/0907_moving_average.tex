\documentclass{article}
\usepackage{kotex}

\begin{document}
\noindent
수열 $\{a_n\}$에 대하여, 주기가 \(p\)인 \(\{a_n\}\)의 \textbf{이동평균}(moving average)은 새로운 수열 $\{b_n\}$을 말합니다.
이때 $b_n$의 값은, \(a_{n-p+1}\), \(a_{n-p+2}\), \(\cdots\), \(a_n\)의 평균으로 정의됩니다;
\[b_n=\frac{a_{n-p+1}+a_{n-p+2}+\cdots+a_n}p\]

\noindent
그러니까 \(p=3\)이면
\(b_3=\frac{a_1+a_2+a_3}3=\frac{0+0+3}3=1\), 
\(b_4=\frac{a_2+a_3+a_4}3=\frac{0+3+3}3=2\)
와 같이 계산됩니다.
이것을 표로 표현해보면

\begin{figure*}[h!]
\centering
\begin{tabular}{c|ccccc ccccc}
$n$		&1&2&3&4&5	&6&7&8&9&10\\\hline
$a_n$	&0&0&3&3&3	&3&6&9&9&12\\	
$b_n$	& & &1&2&3	&3&4&6&8&10\\	
\end{tabular}
\end{figure*}

\noindent
와 같이 표현됩니다.
따라서, 수열 \(\{b_n\}\)은 수열 \(\{a_n\}\)을 뒤에서 따라가는 형태가 됩니다.
이 이동평균의 개념은 주식 예측문제에서 많이 쓰입니다.

\end{document}